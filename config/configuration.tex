\usepackage[utf8]{inputenc}
\usepackage[T1]{fontenc}

\usepackage[left=2.5cm,right=2.5cm,top=2cm,bottom=3cm]{geometry}%réglages des marges du document selon vos préférences ou celles de votre établissemant
\setlength{\headheight}{15pt}% hauteur de l'entête

\usepackage{array}  %pour les array et binomes de newton 
\usepackage{amsmath,amsfonts,amssymb}%extensions de l'ams pour les mathématiques
\usepackage{amsthm} %pour les théoremes
\usepackage{colortbl}
\usepackage[dvipsnames]{xcolor}
\usepackage{comment} % pour les commentaires
\usepackage{dsfont} %fonction indicatrice
\usepackage{fancybox} %pour shadow box
\usepackage[Rejne]{fncychap}%pour de jolis titres de chapitres voir la doc pour d'autres styles.
\usepackage{fancyhdr}%pour les entêtes et pieds de pages
\usepackage{graphicx}%pour insérer images et pdf entre autres
\graphicspath{{images/}}%pour spécifier le chemin d'accès aux images
\usepackage{lmodern}	%celui ci et le suivant pour les boites
\usepackage{shorttoc}%pour la réalisation d'un sommaire.
\usepackage[most]{tcolorbox}

\usepackage{lipsum} %to test paragraphs
\usepackage{verbatim} %for writing as code
\usepackage{fancyvrb} %this is for reducing size of verbatim
\usepackage{subfig} %for having the figures next to each other
\usepackage{stmaryrd} % pacakge for double brackets, notation for integers.

% packages for tables rotated. and algorithms.
\usepackage[ruled,vlined]{algorithm2e}
\usepackage{lscape}
\usepackage{rotating}




\usepackage[english]{babel}%pour un document en englais
\usepackage{hyperref}%rend actif les liens, références croisée, toc, ...
		\hypersetup{colorlinks,%
		citecolor=black,%
		filecolor=black,%
		linkcolor=gray,%
		urlcolor=blue} 




		
		
		
%%%%%%%%%%%%%%%%%%%%%%%%%%%%%%%%%%%%%%%%%%%%%%%%%%%%%%%%%%%%%%%%%%%%%%%%%%%%%%%%%%%%%%%%
\makeatletter
\newenvironment{abstract}{%
    \cleardoublepage
    \null\vfil
    \@beginparpenalty\@lowpenalty
    \begin{center}%
      \bfseries \abstractname
      \@endparpenalty\@M
    \end{center}}%
   {\par\vfil\null}
\makeatother


\newenvironment{acknowledgements}{
\renewcommand\abstractname{Acknowledgements}
\begin{abstract}} {\end{abstract}
}


		

\ifColorfulFormat
    \input{../config/config_boxes_colorful}
\else
    \numberwithin{equation}{section}
    \newtheorem{theorem}{Theorem}[section]
    \newtheorem{corollary}[theorem]{Corollary}
    \newtheorem{lemma}[theorem]{Lemma}
    \newtheorem{proposition}[theorem]{Proposition}
    \newtheorem{conjecture}[theorem]{Conjecture}

    \theoremstyle{definition}
    \newtheorem{definition}[theorem]{Definition}
    \newtheorem{remark}[theorem]{Remark}
    \newtheorem{assumption}[theorem]{Assumption}
\fi % Outer \fi
%%%%%%%%%%%%%% change fontsize and properties of titles	%%%%%%%%%%%%%%%%%%%%%%%%%%%%%%%%%%


\ifColorfulFormat

	\usepackage{titlesec} %pour redéfinir les headers

	\newcommand{\sectionbreak}{\clearpage}

	\titleformat{\section}
	{\Large\bfseries\color[RGB]{10, 80, 144}}
	{\textcolor[RGB]{10, 80, 144}{~ \thesection}}
	{1em}{}

	\titleformat{\subsection}
	{\large\bfseries}
	{\rlap{\color[RGB]{238,243,252}\rule[-1.25ex]{\textwidth}{4ex}}\textcolor{Black}{~ \thesubsection}}
	{1em}{}
	% Alice blue (240, 248, 255)

	\titleformat*{\subsubsection}{\large\bfseries}
	\titleformat*{\paragraph}{\large\bfseries}
	\titleformat*{\subparagraph}{\large\bfseries}


	\titlespacing*{\subsection}
	{0pt}{10mm}{7mm}

\fi % Outer \fi








%%%%%%%%%%%%%%%%%%%style front%%%%%%%%%%%%%%%%%%%%%%%%%%%%%%%%%%%%%%%%%	
	\fancypagestyle{front}{%
  		\fancyhf{}%on vide les entêtes
  		\fancyfoot[C]{page \thepage}%
  		\renewcommand{\headrulewidth}{0pt}%trait horizontal pour l'entête
  		\renewcommand{\footrulewidth}{0.4pt}%trait horizontal pour les pieds de pages
  		}


%%%%%%%%%%%%%%%%%%%style main%%%%%%%%%%%%%%%%%%%%%%%%%%%%%%%%%%%%
	\fancypagestyle{main}{%
		\fancyhf{}
  		\renewcommand{\chaptermark}[1]{\markboth{\chaptername\ \thechapter.\ ##1}{}}% redéfintion pour avoir ici les titres des chapitres des sections en minuscules
  		\renewcommand{\sectionmark}[1]{\markright{\thesection\ ##1}}
		\fancyhead[c]{}
		\fancyhead[RO,LE]{\rightmark}%
  		\fancyhead[LO,RE]{\leftmark}
		\fancyfoot[C]{}
		\fancyfoot[RO,LE]{page \thepage}%
  		\fancyfoot[LO,RE]{contact : niels.carioukotlarek@yahoo.com}
  		}


%%%%%%%%%%%%%%%%%%%style back%%%%%%%%%%%%%%%%%%%%%%%%%%%%%%%%%%%%%%%%%	
	\fancypagestyle{back}{%
  		\fancyhf{}%on vide les entêtes
  		\fancyfoot[C]{page \thepage}%
  		\renewcommand{\headrulewidth}{0pt}%trait horizontal pour l'entête
  		\renewcommand{\footrulewidth}{0.4pt}%trait horizontal pour les pieds de pages
		}






\input{../config/config_colors}














%%%%%%%%%%%%%%%%%%%%%%%%%%% TODO
\usepackage{xargs}                      % Use more than one optional parameter in a new commands
\usepackage[colorinlistoftodos,prependcaption,textsize=footnotesize]{todonotes}

%red
\newcommandx{\willdo}[1]{\todo[linecolor=orangebrulee,backgroundcolor=orangebrulee!35,bordercolor=orangebrulee,inline]{#1}}
%green
\newcommandx{\willprecise}[1]{\todo[linecolor=blue,backgroundcolor=blue!25,bordercolor=blue,inline]{#1}}
%purple
\newcommandx{\willimage}[1]{\todo[linecolor=OliveGreen,backgroundcolor=OliveGreen!25,bordercolor=OliveGreen,inline]{#1}}

\newcommandx{\willlastcheck}[1]{\todo[linecolor=SpringGreen,backgroundcolor=SpringGreen!25,bordercolor=SpringGreen,inline]{#1}}
\newcommandx{\willmuchlater}[1]{\todo[linecolor=Plum,backgroundcolor=Plum!25,bordercolor=Plum,inline]{#1}}

\newcommandx{\thiswillnotshow}[1]{\todo[disable,inline]{#1}}
%%%%%%%%%%%%%%%%%%%%%%%%%%%%%%%%%%%%%%
















%%%%%%%%%%%%%%%%%%%%%%%%%
%%%%% Maths Symbols %%%%%
%%%%%%%%%%%%%%%%%%%%%%%%%


%declare operator%https://tex.stackexchange.com/questions/67506/newcommand-vs-declaremathoperator

%\newcommand{•}{•}
%\renewcommand{•}{•}




%---- Ensemles : entiers, reels, complexes... ----
\newcommand{\N}{\mathbb{N}_{\geq 0}}
\newcommand{\Z}{\mathbb{Z}}
\newcommand{\Q}{\mathbb{Q}}
\newcommand{\R}{\mathbb{R}}
\newcommand{\C}{\mathbb{C}}
\newcommand{\Part}{\mathcal{P}}

\newcommand{\abs}[1]{\lvert #1\rvert} 
\newcommand{\norm}[1]{\lVert #1\rVert}
\newcommand{\braket}[1]{\left\langle #1 \right\rvert}
\DeclareMathOperator{\MSE}{MSE}
\DeclareMathOperator{\MISE}{MISE}
\DeclareMathOperator{\sign}{Sign} 

%\newcommand{\systeme}[1][2]{ \left\{ \begin{array}{#1}#2\end{array} \right. }}

\newcommand{\Tau}{\mathcal{T}} % big tau


\newcommand{\Mat}{\mathrm{Mat}}
\DeclareMathOperator{\fix}{fix}
\DeclareMathOperator{\End}{End}        %Endomorphismes
\DeclareMathOperator{\Hom}{Hom}
\DeclareMathOperator{\Id}{Id}
\DeclareMathOperator{\image}{image}
\DeclareMathOperator{\im}{Im}
\DeclareMathOperator{\tr}{tr}
\DeclareMathOperator{\Tr}{Tr}
\newcommand{\Bilin}{\mathrm{Bilin}}
\newcommand{\Vect}{\mathrm{Vect}}
\newcommand{\Ker}{\mathop{\mathrm{Ker}}\nolimits}
\newcommand{\rg}{\mathop{\mathrm{rg}}\nolimits}
\newcommand{\Sp}{\mathsf{Sp}}
\newcommand{\dx}{\partial_x}                    
\newcommand{\dy}{\partial_y}


%fonction charactéristique
\DeclareMathOperator{\11charac}{\mathds{1}}

%arg max and min 
\DeclareMathOperator*{\argmax}{\arg\!\max}
\DeclareMathOperator*{\argmin}{\arg\!\min}

%probability :
\newcommand{\E}{ \mathbb E }
\newcommand{\Var}{\mathrm{Var}}
\newcommand{\Cov}{\mathrm{Cov}}




%---- Opérateurs utiles ----

\newcommand*\divise{\mathrel{\mid}}
\renewcommand{\bigoplus}{\mathop{\hbox{\large $\oplus$}}}
\newcommand*{\Bigoplus}{\mathop{\hbox{\Large $\oplus$}}}

\DeclareMathOperator{\Non}{non} 

\newcommand{\pgcd}{\text{pgcd}}
\newcommand{\ppcm}{\text{ppcm}}
\DeclareMathOperator*{\Card}{Card}
\DeclareMathOperator*{\supp}{supp}


\newcommand*\conj[1]{%
   \hbox{%
     \vbox{%
       \hrule height 0.5pt % The actual bar
       \kern0.5ex%         % Distance between bar and symbol
       \hbox{%
         \kern-0.1em%      % Shortening on the left side
         \ensuremath{#1}%
         \kern-0.1em%      % Shortening on the right side
       }%
     }%
   }%
} 


\newcommand*\mean[1]{%
   \hbox{%
     \vbox{%
       \hrule height 0.5pt % The actual bar
       \kern0.5ex%         % Distance between bar and symbol
       \hbox{%
         \kern-0.1em%      % Shortening on the left side
         \ensuremath{#1}%
         \kern-0.1em%      % Shortening on the right side
       }%
     }%
   }%
} 



%% eviter les boucles infinies :
\let\oldforall\forall
\renewcommand{\forall}{\oldforall \, }

\let\oldexist\exists
\renewcommand{\exists}{\oldexist \: }

\newcommand\existu{\oldexist! \: }












%---- Démonstrations ----
\newcommand{\sensdirect}{\par\smallbreak\mbox{$(\implies)$}\xspace}
\newcommand{\sensindirect}{\par\smallbreak\mbox{$(\impliedby)$}\xspace}
\newcommand{\implication}[2]{\par\smallbreak\mbox{(\romannumeral#1)$\implies$(\romannumeral#2)}}

\newcommand\analyse{\mbox{\sl \textbf{Analyse: }}\xspace}
\newcommand\synthese{\mbox{\sl \textbf{Synthèse: }}\xspace}
\newcommand\conclusion{\mbox{\sl \textbf{Conclusion: }}\xspace}

\newcommand\existence{\mbox{\sl \textbf{Existence: }}\xspace}
\newcommand\unicite{\mbox{\sl \textbf{Unicité:}}\xspace}


\newcommand\casparticulier{\mbox{\sl \textbf{Cas particulier: }}\xspace}
\newcommand\heredite{\mbox{\sl \textbf{Hérédité: }}\xspace}




\newcommand\cas[1]{\mbox{{\sl \textbf{Cas:}\xspace #1.}}\xspace}
\newcommand\casgeneral{\mbox{\sl \textbf{Cas général:}}\xspace}








% do not trigger any warning related to hbox. I.E. that the line is too long. I had plenty of such useless warnings. 
\hbadness = 15000
\hfuzz = 100 pt 
\vbadness=\maxdimen